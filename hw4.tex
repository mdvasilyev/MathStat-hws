\documentclass{assignment}


\usepackage[utf8]{inputenc}
\usepackage[english,russian]{babel}
\usepackage[T2A]{fontenc}
\usepackage{amsmath}
\usepackage{mathtools}
\usepackage{amsfonts}
\usepackage{listings}
\usepackage{amssymb}
\usepackage{fullpage}
\usepackage{listings}
\usepackage{color}
\usepackage{xcolor}
\usepackage{graphicx}
\usepackage{caption}
\usepackage{qtree}
% \usepackage{minted}
\usepackage{tikz}
\usepackage[export]{adjustbox}
\usepackage{booktabs,makecell}
\usepackage{diagbox}

\usepackage[unicode=true, colorlinks=true, linkcolor=blue, urlcolor=blue]{hyperref} 

\begin{document}

\assignmentTitle{Васильев М.Д. ИТМО}{M4140}{assets/mse_logo.png}{Домашнее задание 4}


\section*{Теоритическая часть}
\begin{enumerate}
    \item \textbf{(2 балла)} C помощью оценки $X_{[1]}$ (то есть центральная статистика должна как-то быть основана на $X_{[1]}$) построить точный доверительный интервал для параметра $\theta$ по выборке объема $n$ из:
    \begin{itemize}
        \item Равномерного распределения на отрезке $[\theta, \theta + 1]$
        \item Равномерного распределения на отрезке $[\theta, 2\theta]$       
    \end{itemize}
    {\color{white}.}
    \start
    Отправлю позже
    \finish
    
    \item \textbf{(1 балл)} Пусть выборка $X_1, \ldots, X_n \sim N(\theta, \theta^2), \theta > 0$. Постройте точный доверительный интервал для параметра $\theta$ уровня доверия $1 - \varepsilon$.
    \start
    Отправлю позже
    \finish
    
    \item \textbf{(1 балл)} В результате проверки 400 электрических лампочек 40 штук оказалось бракованными. Найти доверительный интервал уровня 0.99 для вероятности брака.
    \start
    Отправлю позже
    \finish
    
    \item \textbf{(1 балл)} В Петроградском районе все жители имеют доход не менее 500 тысяч рублей в месяц. Выборочное обследование доходов 200 человек дало средний доход 590 тысяч рублей. В предположении, что случайная величина дохода имеет распределение Парето вида 
    $$F(x)=
    \begin{cases}
        \displaystyle 1 - \frac{x}{x_0}^{-\beta} &, x \geq x_0 \\
        0 &, \texttt{otherwise} 
    \end{cases}    
    $$
    
    где $x_0 = 500$ тысяч рублей, построить доверительный интервал для парамтера $\beta$ с надежностью $90\%$.
    \start
    Отправлю позже
    \finish
\end{enumerate}

\newpage
\section*{Практическая часть}

\begin{enumerate}
    \item \textbf{(3 балла)} Данное задание выполняется по вариантам. Свой номер вы можете увидеть в табличке с баллами. Список заданий доступен по \href{https://drive.google.com/file/d/1AUE3md4TKo0PjZY9W2LFU3uLOUjxeQqI}{ссылке}.
    
    \textit{Замечание: стоит выполнять толькой свое задание, за него будут ставиться баллы. Однако если интересно – можете посмотреть и другие.}
    
    \textbf{Требования:}
    \begin{itemize}
        \item Для сдачи домашнего задания используйте \texttt{Google Colab/Github}. \textit{Ожидается, что ноутбук будет выложен на \texttt{Github/Colab} (не файл \texttt{.ipynb}, присланный куда-либо)}.
        \item Для данного домашнего задания можно использовать языки программирования \texttt{R, Python}.
        \item Задания необходимо выполнять в \texttt{RMarkdown} либо \texttt{Jupyter Notebook} \textit{с комментариями и пояснениями}.
    \end{itemize}
    {\color{white}.}
    \start
    Отправлю позже
    \finish
\end{enumerate}




\end{document}
