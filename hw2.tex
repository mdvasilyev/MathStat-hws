\documentclass{assignment}


\usepackage[utf8]{inputenc}
\usepackage[english,russian]{babel}
\usepackage[T2A]{fontenc}
\usepackage{amsmath}
\usepackage{mathtools}
\usepackage{amsfonts}
\usepackage{listings}
\usepackage{amssymb}
\usepackage{fullpage}
\usepackage{listings}
\usepackage{color}
\usepackage{xcolor}
\usepackage{graphicx}
\usepackage{caption}
\usepackage{qtree}
% \usepackage{minted}
\usepackage{tikz}
\usepackage[export]{adjustbox}
\usepackage{booktabs,makecell}
\usepackage{diagbox}

\usepackage[unicode=true, colorlinks=true, linkcolor=blue, urlcolor=blue]{hyperref} 

\begin{document}

\assignmentTitle{Васильев М.Д. ИТМО}{M4140}{assets/mse_logo.png}{Домашнее задание 2}

\section*{Теоритическая часть}
\begin{enumerate}
    \item \textbf{(1 балл)} $Y_1, \ldots, Y_n$ – выборка с распределением Парето:
    $$
    F(x) = 
    \begin{cases}
        1 - x^{-\alpha}, & x \geq 1\\
        0, & x < 1
    \end{cases}
    $$

    Оцените параметр $\alpha$ с помощью метода моментов. При всех ли значениях параметра $\alpha$ это возможно?
    \start
    Найдем плотность распределения:
    \begin{eqnarray}
        &F(x) = 1 - x^{-\alpha} \\
        &F(x) = 1 - (\frac{x_m}{x})^{k} \\ 
        &x_m = 1 \\
        &k = \alpha \\
        &f(x) = \frac{k x_m^k}{x^{k+1}} = \frac{\alpha}{x^{\alpha+1}}
    \end{eqnarray}
    Найдем математическое ожидание
    \begin{eqnarray}
        \overline{Y} = \mathbb{E}Y_i = \int_{1}^{\infty} x \frac{\alpha}{x^{\alpha+1}} dx = \int_{1}^{\infty} \frac{\alpha}{x^{\alpha}} dx = \frac{\alpha}{1-\alpha} x^{1-a}|_1^\infty = \\
        \frac{\alpha}{1-\alpha} (\infty^{1-a} - 1^{1-a}) = \frac{\alpha}{1-\alpha} (\infty^{1-a} - 1)
    \end{eqnarray}
    Сходится к $\frac{\alpha}{\alpha-1}$ при $1-\alpha<0$.
    \begin{eqnarray}
        &\frac{\alpha}{\alpha-1} = \overline{Y} \\
        &\alpha = \alpha\overline{Y} - \overline{Y} \\
        &\alpha(1-\overline{Y}) = -\overline{Y} \\
        &\alpha = \frac{\overline{Y}}{(\overline{Y}-1)} \\
        &\alpha > 1
    \end{eqnarray}
    \finish

    \item \textbf{(2 балла)} Пусть у нас есть выборка $X_1, \ldots, X_n, a \in \mathbb{R}, b \geq 0$

        \begin{table}[!ht]
        \centering
        \caption{Распределение $X_n$}
        \begin{tabular}{ c c c } 
        \toprule
        $X_n$ & \makecell{$e^{-an}$} & \makecell{$e^{an}$} \\ 
        \midrule
        $P$ &  $1 - e^{-bn}$ & $e^{-bn}$ \\
        \bottomrule
        \end{tabular}
        \end{table}
        При каких $a$ и $b$ $X_{n} \xrightarrow{P} 0$?

    \item \textbf{(1 балл)} Имеется выборка $\displaystyle X_1, \ldots, X_n \sim \texttt{Geom}(p), p = \frac{1}{1+\theta}, \theta \geq 0$. Найдите оценку для $\theta$ методом моментов и докажите ее несмещенность.

    \item \textbf{(2 балла)} Методом выборочных характеристик найдите оценку для параметра $k$ распределения $\chi^2(k)$. Докажите ее несмещенность и посчитайте дисперсию оценки.

    
\newpage

\end{enumerate}

\end{document}
