\documentclass{assignment}
\usepackage[utf8]{inputenc}
\usepackage[russian]{babel}

\begin{document}

\assignmentTitle{ФИО}{Номер}{assets/mse_logo.png}{Математическая статистика}{Домашнее задание 1}

\section*{Теоритическая часть}
\begin{enumerate}
    \item \textbf{(1 балл)} Пусть $X_1, \ldots, X_n \sim $ \texttt{Bernoulli(p)}. Какое распределение имеет выборка $Y_1, \ldots, Y_n$, где $\forall i~:~Y_i = F(X_i)$, а $F(t)$ -- функция распределения \texttt{Bernoulli}?

    \item \textbf{(1 балл)} Пусть $F_n(t)$ -- эмпирическая функция распределения, построенная по выборке $X_1, \ldots, X_n$. Является ли эмпирической функцией распределения функция:
    \begin{itemize}
        \item $F_n(t^3)$
        \item $(F_n(t))^3$
    \end{itemize}
    Если да, то предъявите выборку, по которой функция была построена.

    \item \textbf{(1 балл)} Дана выборка $X_1, \ldots, X_n \sim $ \texttt{Bernoulli(p)}. Проверить несмещенность и состоятельность для следующих оценок:
    \begin{itemize}
        \item $X_1$ для параметра $p$
        \item $X_1X_2$ для параметра $p^2$
        \item $X_1(1 - X_2)$ для параметра $p(1 - p)$
    \end{itemize}

    \item \textbf{(1 балл)} Дана выборка $X_1, \ldots, X_n \sim $ \texttt{U[0, a]}, $a > 1$. По ней построена эмпирическая функция распределения $F_n(t)$. Для какого параметра $\theta = \theta(a)$ статистика $F_n(1)$ является несмещенной оценкой? Является ли она состоятельной оценкой того же параметра?
\end{enumerate}

\newpage

\section*{Практическая часть}
\begin{enumerate}
    \item \textbf{(3 балла)} Эмпирическая функция распределения на самом деле является более точной оценкой генеральной функции распределения, нежели просто состоятельная оценка значения в точке. А именно, имеет место теорема Гливенко-Кантелли:
    $$\sup_x |F(x) - \bar{F}_n(x)| \to 0, n \to \infty$$

    Доказывать подобного рода утверждения в нашем курсе времени нет, однако этот факт достаточно несложно увидеть прямым моделированием. Конечно, исследовать сходимость “почти наверное” моделированием не очень удобно, но мы можем хотя бы рассмотреть более слабую сходимость по вероятности, которая напрямую следует из сходимости “почти наверное”. А для установления сходимости по вероятности у нас есть удобные инструменты: например, всевозможные неравенства, связывающие сходимость по вероятности со сходимостью выборочных характеристик.
    Итак, прямым моделированием покажите выполнение теоремы Гливенко-Кантелли (в “слабом смысле”: со сходимостью по вероятности к нулю). 

    \textbf{Требования:}
    \begin{itemize}
        \item Для сдачи домашнего задания используйте \texttt{Google Colab/Github}. \textit{Ожидается, что ноутбук будет выложен на \texttt{Github/Colab} (не файл \texttt{.ipynb}, присланный куда-либо)}.
         \item Для данного домашнего задания можно использовать языки программирования \texttt{R, Python}.
         \item Задания необходимо выполнять в \texttt{RMarkdown} либо \texttt{Jupyter Notebook} \textit{с комментариями и пояснениями}.
    \end{itemize}

\end{enumerate}

\end{document}
