\documentclass{assignment}


\usepackage[utf8]{inputenc}
\usepackage[english,russian]{babel}
\usepackage[T2A]{fontenc}
\usepackage{amsmath}
\usepackage{mathtools}
\usepackage{amsfonts}
\usepackage{listings}
\usepackage{amssymb}
\usepackage{fullpage}
\usepackage{listings}
\usepackage{color}
\usepackage{xcolor}
\usepackage{graphicx}
\usepackage{caption}
\usepackage{qtree}
% \usepackage{minted}
\usepackage{tikz}
\usepackage[export]{adjustbox}
\usepackage{booktabs,makecell}
\usepackage{diagbox}

\usepackage[unicode=true, colorlinks=true, linkcolor=blue, urlcolor=blue]{hyperref} 

\begin{document}

\assignmentTitle{Васильев М.Д. ИТМО}{M4140}{assets/mse_logo.png}{Домашнее задание 3}

\section*{Теоритическая часть}
\begin{enumerate}
    \item \textbf{(2 балла)} Найти оценки методом максимального правдоподобия параметра $\theta > 0$, если распределение выборки имеет плотность:
    \begin{itemize}
        \item $\theta t^{\theta - 1}, \quad t \in [0, 1]$
        \start
        \begin{align*}
            &L = \prod_{i=1}^{n} \theta t_i^{\theta - 1} = \theta^n \prod_{i=1}^{n} t_i^{\theta - 1} \\
            &\ln L = \ln (\theta^n \prod_{i=1}^{n} t_i^{\theta - 1}) = \ln \theta^n + \ln (\prod_{i=1}^{n} t_i^{\theta - 1}) = n \ln \theta + \sum_{i=1}^{n} \ln (t_i^{\theta - 1}) = \\
            &n \ln \theta + (\theta - 1)\sum_{i=1}^{n} \ln (t_i) \\
            &0 = \frac{\partial}{\partial \theta}\ln L = \frac{\partial}{\partial \theta}(n \ln \theta + (\theta - 1)\sum_{i=1}^{n} \ln (t_i)) = \frac{n}{\theta} + \sum_{i=1}^{n}\ln (t_i) \\
            &\frac{n}{\hat{\theta}} = -\sum_{i=1}^{n}\ln (t_i) \\
            & \hat{\theta} = - \frac{n}{\sum_{i=1}^{n}\ln (t_i)}
        \end{align*}
        \finish
        \item $\displaystyle \frac{2t}{\theta^2},\quad \quad  t \in [0, \theta]$
        \item $\displaystyle \frac{\theta e^{\frac{-\theta^2}{2t}}}{\sqrt{2 \pi t^3}}, \quad t > 0$
        \item $\displaystyle \frac{\theta (\ln{t})^{\theta - 1}}{t}, \quad t \in [1, e]$
    \end{itemize}

    \item \textbf{(2 балла)} Пусть выборка $X_1, \ldots, X_n$ из  распределения Коши с параметром $\gamma$. 
    Плотность задается следующим образом:
    $$
    p_{\gamma}(x) = \frac{1}{\pi \big(1 + (x - \gamma)^2\big)}
    $$

    Существует ли оценка методом максимального правдоподобия для параметра $\gamma$ в общем случае? Поясните.

    Попробуйте найти оценку ММП по выборке из двух наблюдений: $X_1 = -2, X_2 = 2$.
\end{enumerate}
    
\end{document}
