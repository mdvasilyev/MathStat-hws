\documentclass{assignment}


\usepackage[utf8]{inputenc}
\usepackage[english,russian]{babel}
\usepackage[T2A]{fontenc}
\usepackage{amsmath}
\usepackage{mathtools}
\usepackage{amsfonts}
\usepackage{listings}
\usepackage{amssymb}
\usepackage{fullpage}
\usepackage{listings}
\usepackage{color}
\usepackage{xcolor}
\usepackage{graphicx}
\usepackage{caption}
\usepackage{qtree}
% \usepackage{minted}
\usepackage{tikz}
\usepackage[export]{adjustbox}
\usepackage{booktabs,makecell}
\usepackage{diagbox}

\usepackage[unicode=true, colorlinks=true, linkcolor=blue, urlcolor=blue]{hyperref} 

\begin{document}

\assignmentTitle{Васильев М.Д. ИТМО}{M4140}{assets/mse_logo.png}{Дополнительне задачи 1}

\begin{enumerate}
    \item \textbf{(1 балл)}  При нейтронной бомбардировке ядер урана начинается расщепление ядра, при котором ядро урана распадается на две части различного типа; в камере Вильсона это явление обнаруживается в виде двух траекторий, исходящих из одной точки. Эти траектории вскоре разделяются на несколько ветвей, получающихся от столкновения частиц с молекулами газа в камере. Можно показать, что число ветвей в одной траектории имеет так называемое «двойное» распределение Пуассона:

    $$P(X = k) = \frac{1}{2}\Big(\frac{\lambda_1^k}{k!}e^{-\lambda_1} + \frac{\lambda_2^k}{k!}e^{-\lambda_2} \Big), k > 0, 1, \ldots,$$

    где $\lambda_1 < \lambda_2$ некоторые положительные постоянные. Используя методов моментов, построить векторную оценку параметра $(\lambda_1, \lambda_2)$.
    
    \item \textbf{(1 балл)} Определить границы значений коэффициента эксцесса. Обосновать ответ.
    \start
    Коэффициент эксцесса вычисляется по формуле
    \begin{equation}
        \gamma_2 = \frac{\mu_4}{\sigma^4} - 3,
    \end{equation}
    где $\mu_4$ — центральный момент четвертого порядка, а $\sigma$ — стандартное отклонение. При $\gamma_2 < 0$ распределение более плоское по сравнению с нормальным, при $\gamma_2 > 0$ распределение имеет более острый пик по сравнению с нормальным, а при $\gamma_2 = 0$ распределение по остроте совпадает с нормальным. Коэффициент эксцесса ограничен снизу:
    \begin{equation}
        \gamma_2 \in [-2, +\infty).
    \end{equation}
    Это обусловлено ограничением на центральный момент, ведь для минимизации $\gamma_2$ необходимо минимизировать $\frac{\mu_4}{\sigma^4}$. А этот одночлен достигает своего минимума при 
    \begin{equation}
        \mu_4 = \sigma^4 \Rightarrow \gamma_2(\mu_4 = \sigma^4) = 1 - 3 = - 2
    \end{equation}
    Сверху значение $\gamma_2$ не ограничено, потому что существуют распределения с очень тяжелыми хвостами, для которых отношение $\frac{\mu_4}{\sigma^4}$ не будет ограничено сверху.
    
    \textbf{Ответ}: $\gamma_2 \in [-2, +\infty)$.
    \finish

    \item \textbf{(1 балл)} Пусть выборка $X_1, \ldots, X_n \sim \texttt{Weibull}$ с функцией распределения $F(x) = 1 - e^{-\lambda x^2}, x \geq 0, \lambda > 0$. Найдите оценку параметра $\lambda$ методом максимального правдоподобия.
    \start
    Определим плотность вероятности:
    \begin{equation}
        f(x) = \frac{d}{dx}F(x) = \frac{d}{dx}(1 - e^{-\lambda x^2}) = 2 x \lambda e^{-\lambda x^2}
    \end{equation}
    Функция правдоподобия:
    \begin{align*}
        &L(\lambda) = \prod_{i=1}^{n}f(X_i) = \prod_{i=1}^{n}2 X_i \lambda e^{-\lambda X_i^2} \\
        &\ln L(\lambda) = \ln (\prod_{i=1}^{n}2 X_i \lambda e^{-\lambda X_i^2}) = \sum_{i=1}^{n} \ln (2 X_i \lambda e^{-\lambda X_i^2}) = \sum_{i=1}^{n} (\ln 2 + \ln X_i  + \ln \lambda + \ln e^{-\lambda X_i^2}) = \\
        &n\ln 2 + \sum_{i=1}^{n}\ln X_i  + n\ln \lambda + \sum_{i=1}^{n}(-\lambda X_i^2) \\
        &\frac{d}{d\lambda}\ln L(\lambda) = \frac{d}{d\lambda} (n\ln 2 + \sum_{i=1}^{n}\ln X_i  + n\ln \lambda + \sum_{i=1}^{n}\ln e^{-\lambda X_i^2}) = \frac{n}{\lambda} - \sum_{i=1}^{n} X_i^2 \\
        &\frac{d}{d\lambda}\ln L(\lambda) = 0 \Leftrightarrow \frac{n}{\lambda} - \sum_{i=1}^{n} X_i^2 = 0 \Rightarrow \hat{\lambda} = \frac{n}{\sum_{i=1}^{n} X_i^2}
    \end{align*}
    
    \textbf{Ответ}: $\hat{\lambda} = \frac{n}{\sum_{i=1}^{n} X_i^2}$.
    \finish

    \item \textbf{(1 балл)} Для отрасли, включающей 1200 фирм, составлена случайная выборка из 19 фирм. По выборке оказалось, что в фирме в среднем работают 77.5 человека при выборочном среднем квадратическом отклонении 25 человек. Пользуясь 95\%-ным доверительным интервалом, оцените среднее число работающих в фирме по всей отрасли и общее число работающих в отрасли. Предполагается, что случайное число работников фирмы описывается нормальным распределением.
    \start
    Для построения доверительного интервала используем t-распределение. Доверительный интервал:
    \begin{equation}
        \bar{x} \pm t_{\alpha/2}\cdot \frac{s}{\sqrt{n}},
    \end{equation}
    где $\bar{x}$ — выборочное среднее, $t_{\alpha/2}$ — t-критическое значение для уровня значимости $\alpha$, $s$ — выборочное стандартное отклонение, $n$ — размер выборки. Для уровня доверия 95\%, $\alpha = 0.05$ и числа степеней свободы $n-1 = 19-1 = 18$ $t_0.025 \approx 2.1$. Тогда доверительный интервал будет равен:
    \begin{equation}
        \bar{x} \pm t_{\alpha/2}\cdot \frac{s}{\sqrt{n}} = 77.5 \pm 2.1\cdot \frac{25}{\sqrt{19}} \approx 77.5 \pm 12 \Rightarrow (65.5, 89.5)
    \end{equation}
    Если всего фирм 1200, то общее число работающих в отрасли можно вычислить так:
    \begin{equation}
        (65.5\cdot 1200, 89.5\cdot 1200) = (78600, 107400).
    \end{equation}

    \textbf{Ответ}: по фирме (65.5, 89.5), по отрасли (78600, 107400).
    \finish

    \item \textbf{(1 балл)} Рекламная компания утверждает, что у 80\% их клиентов выросли продажи. Случайная выборка из 100 клиентов показала, что у 77 из них выросли продажи. Проверьте на уровне значимости в 5\% гипотезу о том, что рекламная компания права против альтернативы, что продажи выросли менее чем у 80\% их клиентов.
    \start
    \begin{align*}
        &H_0: p=0.8 \\
        &H_1: p<0.8
    \end{align*}
    Рассчитаем z-статистику:
    \begin{equation}
        z = \frac{\hat{p} - p_0}{\sqrt{\frac{p_0(1-p_0)}{n}}} = \frac{0.77 - 0.8}{\sqrt{\frac{0.8(1-0.8)}{100}}} = -0.75
    \end{equation}
    Для $\alpha = 0.05$ критическое значение $z_{cr} \approx -1.645$. В нашем случае $z > z_{cr}$, поэтому нельзя отвергать нулевую гипотезу на уровне значимости 0.05.
    
    \textbf{Ответ}: недостаточно оснований полагать, что рекламная компания говорит неправду.
    \finish

    \item \textbf{(1 балл)} Страховая компания хочет изучить шансы попасть в ДТП среди трех групп: молодых, пожилых водителей и водителей среднего возраста. 10\% из 50 молодых людей, 10\% из 40 пожилых людей и 15\% из 60 людей среднего возраста в год попадают в ДТП. Проверьте на уровне значимости 1\%, гипотезу о том, что частота попадания в ДТП не зависит от возраста.
    \start
    \begin{align*}
        &H_0: \text{Частота попадания в ДТП не зависит от возраста} \\
        &H_1: \text{Частота попадания в ДТП зависит от возраста}
    \end{align*}
    Таблица частот ДТП:
    \begin{table}[h!]
        \begin{tabular}{|l|l|l|l|}
        \hline
        группа            & попадают в ДТП & не попадают в ДТП & всего \\ \hline
        молодые           & 5              & 45                & 50    \\ \hline
        пожилые           & 4              & 36                & 40    \\ \hline
        среднего возраста & 9              & 51                & 60    \\ \hline
        всего             & 18             & 132               & 150   \\ \hline
        \end{tabular}
    \end{table}
    
    Ожидаемые частоты:
    \begin{equation}
        E_{ij} = \frac{sum\_row_i\cdot sum\_col_j}{total}
    \end{equation}
    \begin{table}[h!]
        \begin{tabular}{|l|l|l|l|}
        \hline
        группа            & попадают в ДТП & не попадают в ДТП & всего \\ \hline
        молодые           & 6              & 44                & 50    \\ \hline
        пожилые           & 4.8              & 35.2                & 40    \\ \hline
        среднего возраста & 7.2              & 52.8                & 60    \\ \hline
        всего             & 18             & 132               & 150   \\ \hline
        \end{tabular}
    \end{table}
    
    Рассчитаем $\chi^2 = \sum \frac{O_{ij}\cdot E_{ij}}{E_{ij}}$, где $O_{ij}$ -- наблюдаемые частоты, а $E_{ij}$ -- ожидаемые частоты:
    \begin{equation*}
        \chi^2 = \frac{(5-6)^2}{6} + \frac{(45-44)^2}{44} + \frac{(4-4.8)^2}{4.8} + \frac{(36-35.2)^2}{35.2} + \frac{(9-7.2)^2}{7.2} + \frac{(51-52.8)^2}{52.8} \approx 0.8522
    \end{equation*}
    Для уровня значимости $\alpha = 0.01$ и степеней свободы $(r-1)(c-1) = (3-1)(2-1) = 2$ критическое значение $\chi^2_{cr} \approx 9.21$. Так как $\chi^2=0.8522$ и $\chi^2 < \chi^2_{cr}$, то нельзя отвергать нулевую гипотезу на уровне значимости 0.01.

    \textbf{Ответ}: недостаточно оснований полагать, что частота попадания в ДТП зависит от возраста.
    \finish

    \item \textbf{(1 балл)} Доказать, что для геометрического распределения с параметром $\displaystyle p = \frac{1}{\gamma}, \gamma > 1$, выборочное среднее является эффективной оценкой параметра $\gamma$.

    $$P(\xi_{geom} = k) = p(1 - p)^{k - 1}$$
    
    \item \textbf{(1 балл)} Доказать, что для биномиального распределения оценка $\displaystyle \hat{p} = \frac{\bar{X}}{N}$ является эффективной оценкой вероятности $p$.

    $$P(\xi_{binom} = k) = \binom{N}{k} p^{k}(1 - p)^{N - k}$$

    \item \textbf{(1 балл)} Доказать, что эмпирическая функция распределения $F_n(x)$ является эффективной оценкой теоретической функции распределения $F(x)$ при каждом $x$.

    \item \textbf{(2 балла)} По выборке $X_1, \ldots, X_n$ в случае биномиального распределения при известном $N$ методом максимального правдоподобия найти оценку параметра $p$. Совпадает ли эта оценка с оценкой, полученной с помощью метода моментов? Исследовать оценку на несмещенность и состоятельность.

    \item \textbf{(2 балла)} Доказать, что не существует эффективных оценок для параметра $\sigma^2$ нормального распределения $N(a, \sigma^2)$.

    \item \textbf{(2 балла)} Известно, что на выполнение дополнительных задач по курсу ``Математическая статистика'' тратится время, состоящее из постоянного периода и случайной задержки, распределенной показательно. Хронометраж затраченного времени за последние несколько лет по выполнению задач для должников показал, что работа занимает в среднем $N$ минут со средним квадратическим отклонением $M$ минут. С помощью метода моментов оценить вероятность того, что все задачи будут решены за час, а также время, за которое задачи будут выполнены на $95\%$. 
    \begin{itemize}
        \item $N = 45, M = 15$
        \item $N = 40, M = 8$
    \end{itemize}


    \item \textbf{(1 балл)} Построить асимптотический доверительный интервал для параметра $p$ с известным числом испытаний $N$ для биномиального распределения.

    \item \textbf{(1 балл)} Построить асимптотический доверительный интервал для параметра $\theta > 0$ нормального распределения $\mathcal{N}(\theta, \theta^6)$.


    \item \textbf{(3 балла)} По выборке из $N$ наблюдений двумерной нормальной случайно велиичны получен выборочный коэффициент корреляции $\hat{corr}$. Построить доверительный интервал для коэффициента корреляции с надежностью $90\%$. Решить задачу, если: 

    \begin{itemize}
        \item $N = 50,~ \hat{corr} = 0.5$
        \item $N = 100, \hat{corr} = 0.7$
        \item $N = 200, \hat{corr} = 0.6$

    \end{itemize}


    \item \textbf{(2 балла)} Пусть случайная величина имеет распределение Пуассона с параметром $\lambda$. Требуется на уровне значимости $\alpha$ проверить нулевую гипотезу $H_0:~ \lambda = \lambda_0$, если альтернативная гипотеза $H_1:~ \lambda = \lambda_1 > \lambda_0$. Построить критерий отношения правдоподобия используя нормальное приближение. Вычислить объем выборки $N$, необходимый для достижения заданных ошибок первого и второго рода $\alpha$ и $\beta$.

    \item \textbf{(2 балла)} Проверяется нулевая гипотеза о том, что случайная величина равномерно распределена на отрезке $[-1, 1]$ против гипотзеы, что она имеет нормальное распределение $\mathcal{N}(0, \sigma^2)$. Построить критерий отношения правдоподобия в общем виде и в частном случае при $\displaystyle n = 2, \alpha < \frac{\pi}{4}$.
    


\end{enumerate}

\end{document}
